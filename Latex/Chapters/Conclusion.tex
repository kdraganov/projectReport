\chapter{Conclusion \& Future Work}

\section{Future Work}
Distinction between incidents and congestion (use peak detection algorithm).
Data from other source could be used (taxis AVL, couriers services AVL) etc.
Historical data could be employed in order to make further analysis and correlations with weather data, time or the day/week/year etc.
Increasing the frequency of the data means that we could make use of the actual geo location (alternative approach to one taken) information in order to calculate and monitor the bus speeds rather than the preprocessed schedule deviation value.

Using data from taxis [Cite - Requirements and Potential of GPS-based Floating Car Data for Traffic Management Stockholm Case study]

\section{Conclusion}
Summary of the project and the report...

\begin{itemize}
	\item What has worked well
	\item What has not worked well
	\item Lessons learnt
\end{itemize}

A project post-mortem, also called a project retrospective, is a process for evaluating the success (or failure) of a project's ability to meet business goals. 

Post-mortems can encompass both quantitative data and qualitative data. Quantitative data include the variance between the hours estimated for a project and the actual hours incurred. Qualitative data will often include stakeholder satisfaction, end-user satisfaction, team satisfaction, potential re usability and perceived quality of end-deliverables.