%Either in a seperate section or throughout the report demonstrate that you are aware of the \textbf{Code of Conduct \& Code of Good Practice} issued by the British Computer Society and have applied their principles, where appropriate, as you carried out your project.
\chapter{Professional \& Ethical Issues}
Throughout every stage of this project I have made every effort to follow the rules and guidelines that are set out by the British Computer Society (BCS) Code of Conduct \& Code of Practice \cite{bcsCodeOfConduct}. These are rules and professional standards that govern the individual decisions and behaviour. The main rules that apply almost to every software development project states the individual should:
\begin{itemize}
	\item "have due regard for public health, privacy, security and wellbeing of others and
the environment."\cite{bcsCodeOfConduct}
	\item "have due regard for the legitimate rights of Third Parties"\cite{bcsCodeOfConduct}.
\end{itemize}

The whole project has been planned, designed and developed with both of these rules, as well as other rules and standards, in mind. The system makes use of a number of third party libraries and framework. However I have explicitly stated the use of any such libraries and provided the according reference to the source of the original idea/product. I have also given references to any work or ideas that I have made use of or have been inspired by throughout the project.

I have also tried to make sure that the applications that were developed as part of this project do not pose any harm neither to the computers they are running on or interacting with, nor to their users. The tool is expected to run 24/7 with a large number of files being processed every day. This means that we need to make sure that it does not contain any memory leaks as discussed in previous chapters. I have also used appropriate method to safeguard the database from any SQL injection \cite{Su2006} which could potentially alter the data unintentionally or without the appropriate permission. However, it should be noted that this is a prototypical system and not a fully working and security-proof production version.

The web application displays the last time and date when the disruption engine has updated its state. This value represent the latest time of data value from the iBud AVL feeds. This allows the users to be aware how old the data they are seeing actually is.

In case of failure during the execution of the disruption engine, an alert system can easily be set. Simple example of such alerting could be done via emails being send out in case the system fails. This will allow for the responsible maintenance personnel to be alerted in time. However, if the system malfunctions in manner such that it continues its execution and continues to generate output, it can lead to confusion among the users of the application. We cannot be responsible if this is caused by the input data. However if the the input is correct and the calculations are wrong it will possibly lead to the loss of trust in the application. In order to minimise such risks thorough testing should be carried out once the system is deployed. It is recommended that the system undergoes trial runs with real data being pushed in real time. This would be another test building on the rest of the testing that has been done to prove the correct system behaviour in real-life scenarios.

%If the system fails and it stops updating the list of disruptions, this will be easy to detect by the normal user as the user interface has last updated time. If this value hangs at a given time and say 10 or more minutes have passed since then either the system is not working or there is problem with the input data (e.g. feeds are not being pushed).

%However if the system calculates abnormal delays and this is not due to the data it could lead to confusion with the CentreComm staff and it can make the sceptical in their future interactions with the system. In order to prevent this from happening we need to make sure the system is fully tested. It also would be beneficial to do trial runs of the system with real data in real time in order to analyse and evaluate its behaviours in real-life scenarios.

%If this systems goes into production possible issues that might arise from malfu
%Are there any issues specific to your domain? For example, what would be the real-world consequences if your system malfunctioned?