\chapter{Conclusion \& Future Work}

\section{Future Work}
Distinction between incidents and congestion (use peak detection algorithm).
Data from other source could be used (taxis AVL, couriers services AVL) etc.
Historical data could be employed in order to make further analysis and correlations with weather data, time or the day/week/year etc.
Increasing the frequency of the data means that we could make use of the actual geo location (alternative approach to one taken) information in order to calculate and monitor the bus speeds rather than the preprocessed schedule deviation value.

Using data from taxis [Cite - Requirements and Potential of GPS-based Floating Car Data for Traffic Management Stockholm Case study]

\section{Conclusion}
This report proposed a prototypical tool for detecting disruptions in London bus network. This is achieved by employing moving average smoothing technique for analysis of the time series data presented. The main strength of this approach is its simplicity. This is growing area of interest for intelligent transportation systems (ITS) and I expect to receive much more attention in near future with the rise in AVL data availability from different sources. I have also given some directions and proposal for improving and driving this work further.

This project has taught me a lot about transportation systems and networks. It has given me great insight into the complex operations that go into operating and managing large bus networks as is London's one. In addition to that I have also learnt a lot about different statistical techniques available for analysis of time series data. Such techniques are very useful and could be applied not only in the domain of this project, but in every problem where there is time series data which needs to be analysed. The software development approach taken for developing the prototype for this project seemed to work well. It has allowed me to gather useful feedback early on and to refine the project requirements as initially the user requirements were very broad and vague. This is probably due to the fact that there is not much closely related work previously done. One thing that did not work so well is the evaluation of the software system. This is mainly owned to delay in planning and carrying out this evaluation. However this is something I have now learnt and would help me better plan my future projects I undertake.

%Spend significant amount of time in discussions and requirments gathering with CentreComm in order to fully undestand the requirments. This was a major part of the project however this resulted in having limited time for implementation, testing and evaluation. More test data need to be obtained and futher evaluations need to be done in order to draw conclusions whether this approach is accurate enough to be relied for an objective and real monitoring alerts. This project however was successful in formalising the requirments from such system and laying out the ground work.

%A project post-mortem, also called a project retrospective, is a process for evaluating the success (or failure) of a project's ability to meet business goals. Post-mortems can encompass both quantitative data and qualitative data. Quantitative data include the variance between the hours estimated for a project and the actual hours incurred. Qualitative data will often include stakeholder satisfaction, end-user satisfaction, team satisfaction, potential re usability and perceived quality of end-deliverables.