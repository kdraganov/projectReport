%You must provide an adequate user guide for your software. The guide should provide easily understood instructions on how to use your software. A particularly useful approach is to treat the user guide as a walk-through of a typical session, or set of sessions, which collectively display all of the features of your package. Technical details of how the package works are rarely required. Keep the guide concise and simple. The extensive use of diagrams, illustrating the package in action, can often be particularly helpful. The user guide is sometimes included as a chapter in the main body of the report, but is often better included in an appendix to the main report.
\chapter{User Guide}

\section{Disruption Engine}
Below are the instructions how to set up and run the disruption engine tool.

\subsection{Installation}
The disruption engine installation consists of the following steps:
\begin{enumerate}
	\item Obtain the jar package of the tool.
	\item Obtain and have all of the above dependencies in the CLASSPATH of the system.
	\item Set up a database using the provided database script at the end of the source code listing.
	\item Create XML file which to contain the connection settings to the respective database.
\end{enumerate}
Below is the list of the required libraries and dependencies:

\begin{itemize}
\item JRE 8 - the library and documentation can be found on \url{http://www.oracle.com/technetwork/java/javase/documentation/index.html}.
\item Scala Library 2.11.5 - \url{http://www.scala-lang.org/news/2.11.5}.
\item Scala XML 2.11-1.0.2 - is used for working with XML files.
\item JDBC PostgreSQL - PostgreSQL 9.0 JDBC3 and PostgreSQL 9.4 JDBC4 are both used which can be found in \url{https://jdbc.postgresql.org/download.html}.
\item SLF4J 1.6.4 - is used for logging. Documentation and the library can be found here \url{http://www.slf4j.org/}. It also requires the bellow two libraries in order to work.
\item Logback 1.0.1 - is used for logging \url{http://logback.qos.ch/}. The engine makes use of the classic and core packages both version 1.0.1.
\item JCoord-1.0 - is used for converting easting/northing locations to the respective longitude/latitude values. The library was obtained from \url{http://www.jstott.me.uk/jcoord/}.
\item ScalaTest 2.2.4 - library was used for testing. Instructions and the library files can be found in \url{http://www.scalatest.org/download}.
\end{itemize}

\subsection{Execution}
In order to run the application you simply need to execute the iBusDisruptionMonitor.jar with the following command from the command line:
\begin{lstlisting}
  java -jar iBusDisruptionMonitor.jar [path]
\end{lstlisting}
In the above command you need to substitute [path] with the path to an XML file containing the connection settings to a database. The XML file should have the following structure:
\begin{lstlisting}[language=XML]
 <?xml version="1.0" encoding="UTF-8"?>
<connection>
    <host>[HOST]</host>
    <port>[PORT]</port>
    <database>[Database name]</database>
    <user>[USERNAME]</user>
    <password>[PASSWORD]</password>
    <maxPoolSize>5</maxPoolSize>
</connection>
\end{lstlisting}
In this XML file you need to substitute everything between the square brackets with the respective values for your configuration. The database creation script has been provided in the source code listing. 

Executing the above command will start the tool, but do make sure you have set up the right configuration settings in the database before running the application.

If required or in case this moves to production it will be best to run the engine as a service. This can be achieved by using Java Service Wrapper \url{http://wrapper.tanukisoftware.com/doc/english/download.jsp}.


\section{Web Application}

The web application is build using Ruby 2.1.5 (\url{https://www.ruby-lang.org/en/}) on Rails 4.2.0 framework (\url{http://rubyonrails.org/}). The testing and simulations have been carried out on WEBrick 1.3.1 server (\url{http://ruby-doc.org/stdlib-2.0/libdoc/webrick/rdoc/WEBrick.html}) within the IntelliJ IDE. The application makes use of the Foundation framework (\url{http://foundation.zurb.com/}) and some other third-part libraries developed for Ruby on Rails. Full list of the dependencies and how to set up the web application can be found in the read-me file inside the archive with the source code submitted as part of this project.

\section{Database}
The database used for this project is PostgreSQL 9.4 \url{https://wiki.postgresql.org/wiki/What\%27s_new_in_PostgreSQL_9.4}. The database creation scripts can be found in the source code listing for this project. In the source code archive that is submitted along this report I have included a dump of the database which includes all bus stops and bus routes from TFL's bus network. This data is from March 2015 and is not up to date as TFL publishes updated information every few weeks.