Automatic Vehicle Location (AVL) systems for bus fleets have been deployed successfully in many cities. They have enabled improved bus fleet management and operation as well as wide range of information for the travelling public. However there is still a lot of area of utilisation of the data that these AVL systems generate and produce. This report explores and analyses the tools and applications currently available at Transport for London (TFL) bus operation unit. The report then proposes a prototypical tool for monitoring the bus delays in real time in the network. This tool offers objective source of processed information to bus operators and control room staff. However further work need to be done in order to place this tool in production environment as urban delay detection is very complex and unpredictable as will the report justify.

%This report explorers and analysis the needs of an advanced delay monitoring system of London buses. I have carried out literature review of the work done in this area. This is followed by a proposal and implementation of a prototypical tool which to meet the requirements of the emergency control room for the London bus operations. The report tries to evaluate what has been achieved and how this can be further developed and improved.