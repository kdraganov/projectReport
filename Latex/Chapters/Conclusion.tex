\chapter{Conclusion}
During the course of this exciting project we have examined carefully the needs of CentreComm for automation of their current work flow which could lead to better operation, management and cut in costs of controlling London's bus network. This has naturally let to a in depth literature review of the related work and approaches that could be undertaken in order to solve the problem posed by our project. This has resulted in the design and implementation of a prototypical tool for detecting bus delays in real time using iBus AVL data.


This report proposed a prototypical tool for detecting disruptions in London bus network. This is achieved by employing moving average smoothing technique for analysis of the time series data presented. The main strength of this approach is its simplicity. This is growing area of interest for intelligent transportation systems (ITS) and I expect to receive much more attention in near future with the rise in AVL data availability from different sources. I have also given some directions and proposal for improving and driving this work further.

This project has taught me a lot about transportation systems and networks. It has given me great insight into the complex operations that go into operating and managing large bus networks as is London's one. In addition to that I have also learnt a lot about different statistical techniques available for analysis of time series data. Such techniques are very useful and can be applied not only in the domain of this project, but in a broad range of problems where there is time series data which needs to be analysed. 

The software development approach taken for developing the prototype for this project seemed to work well. It has allowed me to gather useful feedback early on and to refine the project requirements as initially the user requirements were very broad and vague. This is probably due to the fact that there is not much closely related work previously done. Significant amount of this project was devoted to gathering and refining the user requirements. This included numerous meetings, discussions and even shadowing at CentreComm. Another major part of this project was carrying out proper background research which even needed to be revisited latter in the project. 

One thing that did not work so well is the evaluation of the software system. This is owned partly due to the lack of readily available information which could be used as well as due to delay in planning and carrying out this evaluation. However this is something I have now learnt and would help me better plan my future projects I undertake.

%As this project addressed new and yet not explored areas of improvement
%This project has examined carefully the needs of CentreComm for automation of their current work flow which could lead to better operation, management and cut in costs of controlling London's bus network.
%It has reviewed the literature for related work and approaches and has resulted in a prototypical tool being designed and implemented. 

\section{Future Work}
The proposed system has provided some initial results as seen in the previous chapter. There is however a lot that could be improved and built upon on it. In this section I provide some suggestions and direction for taking this work further.

One simple extension that could take this project one step ahead is to try to implement a peak/valley detection algorithm which could distinguish if given detected delay/disruption is incident related (very sudden spike) or it is general traffic jam (gradual delay build-up) due to rush hour for instance. This was briefly discussed in chapter 2 of this report however due to limited time of this project it has not been explored in more depth or considered for design and implementation.

Another feature that has received a lot of positive feedback from different levels in TFL is the ability to generate complex historical reports of what has been the network state. According to TFL this would provide more objective view of what has happened in the transport network and who should take the responsibility. 

More historical data could also be employed and correlated with the real-time data received. Examples of such data may include weather data, time of the day/week, workdays compared to weekends and public holidays etc. This could improve the accuracy of the system as well as bring to light some persisting problems under given environments.

The increased popularity and usage of various AVL systems being used by different fleets could be used along with the data present in this report. This could include taxi fleets \cite{rahmani2010requirements}, delivery service fleet, emergency services and many more. Having more information and especially from different sources could result in one central congestion monitoring system which could more accurately and in shorter-terms calculate and predict traffic conditions in the arterial road network of the city. There is the potential for employing data from GPS enabled devices that most people of the general travelling public own these days. This something is already being investigated \cite{thiagarajan2010cooperative} and I can see this will get more attention in the near future.

In addition to everything mentioned above further research and evaluation of the available techniques and data is recommended in order to veirfy that this is the optimal solution to the problem being addressed by this project. Some alternative time series analysis tools and methods could be analysed including Kalman filtering \cite{kalmanFiltering} \cite{Guo201450}, Markov Chains \cite{Qi201495} \cite{Ramezani20121576}, Machine learning \cite{herring2010real}, Bayesian networks \cite{Wang201479}.

%However it is worth to note that spike detection and analysis could be used to classify events that are detected. A simple example could be to distinguish if given detected congestion/disruption is incident related (e.g. sudden) or it is general traffic jam (e.g. rush hour).
%[MOVE THIS TO CONCLUSION AS DIRECTION FOR FUTURE WOKR] - check this http://www.cs.technion.ac.il/users/wwwb/cgi-bin/tr-get.cgi/2012/CS/CS-2012-06.pdf
%Other approaches that could be used include

%Increasing the frequency of the data means that we could make use of the actual geo location (alternative approach to one taken) information in order to calculate and monitor the bus speeds rather than the preprocessed schedule deviation value.

%Spend significant amount of time in discussions and requirments gathering with CentreComm in order to fully undestand the requirments. This was a major part of the project however this resulted in having limited time for implementation, testing and evaluation. More test data need to be obtained and futher evaluations need to be done in order to draw conclusions whether this approach is accurate enough to be relied for an objective and real monitoring alerts. This project however was successful in formalising the requirments from such system and laying out the ground work.

%A project post-mortem, also called a project retrospective, is a process for evaluating the success (or failure) of a project's ability to meet business goals. Post-mortems can encompass both quantitative data and qualitative data. Quantitative data include the variance between the hours estimated for a project and the actual hours incurred. Qualitative data will often include stakeholder satisfaction, end-user satisfaction, team satisfaction, potential re usability and perceived quality of end-deliverables.