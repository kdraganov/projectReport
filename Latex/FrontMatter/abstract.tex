Automatic Vehicle Location systems for bus fleets have been deployed successfully in many cities. They have enabled improved bus fleet management and operation as well as wide range of information for the travelling public. However, there are still processes that can be improved or automated by utilising the data made available by the different systems including the Automatic Vehicle Location one. This report explores and analyses the tools and applications currently available at Transport for London's bus emergency command and control unit. The report then proposes a prototypical tool for monitoring the bus delays in real time in the network. This tool offers an objective source of processed information to bus operators and control room staff. However, further work needs to be done in order for the system to be placed in production environment and relied upon. This is because arterial urban delay detection is very complex and unpredictable as the report will justify. The report concludes with some suggestions of how this project can be improved and driven further.

%In this report, we explore the needs for a real-time disruption monitoring system for London's bus network. Then we propose and develop a prototypical system which to satisfy those requirements. Our approach is based on employing time series analysis techniques in order to detect the network disruptions. We also present the potential benefits of such tools and how it can be further improved and developed.

%\begin{itemize}
%\item Why is this project done?
%\item What problem is the report trying to better understand or solve?
%\item What is the scope of the project (understand the needs, propose a prototype)
%\item What is the main claim? (there is a lot of benefits and potential in this work, however the data provided doesn't satisfy our needs)
%\item Why is it important?
%\item Why should you read the report?
%\end{itemize}


%The report explorers the advancements and developments in the field of Intelligent Transportation Systems (ITS) from which we gather inspiration and ideas. This also helps us give suggestions for further development of the proposed system and future work.

%Automatic Vehicle Location (AVL) systems for bus fleets have been deployed successfully in many cities. They have enabled improved bus fleet management and operation as well as wide range of information for the travelling public. However there is still a lot of area of utilisation of the data that these AVL systems generate and produce. This report explores and analyses the tools and applications currently available at Transport for London (TFL) bus operation unit. The report then proposes a prototypical tool for monitoring the bus delays in real time in the network. This tool offers objective source of processed information to bus operators and control room staff. However further work need to be done in order to place this tool in production environment as urban delay detection is very complex and unpredictable as will the report justify.

%This report explorers and analysis the needs of an advanced delay monitoring system of London buses. I have carried out literature review of the work done in this area. This is followed by a proposal and implementation of a prototypical tool which to meet the requirements of the emergency control room for the London bus operations. The report tries to evaluate what has been achieved and how this can be further developed and improved.