\chapter{Introduction}
%This is one of the most important components of the report. It should begin with a clear statement of what the project is about so that the nature and scope of the project can be understood by a lay reader. It should summarise everything that you set out to achieve, provide a clear summary of the project's background and relevance to other work, and give pointers to the remaining sections of the report, which will contain the bulk of the technical material \cite{einstein}.
\section{Motivation}
London bus network is one of the largest and most advanced bus networks in the worlds. It is responsible for more than 2.4 billion passenger journeys a year \cite{TFL1}. The constant population growth of England's capital has been also driving the expansion and improvement of the transport networks across the city. Transport for London (TFL) is in charge of its operation and its bus network if recognised as one of the top in the world in terms or reliability, affordability and cost-effectiveness \cite{TFL1}. The capital's bus network is continually increasing along with the city's population \cite{TFL2}. 
Maintaining such a large scale network requires careful planing and monitoring. Being able to maintain such high reliability service 24/7 364 days in the year requires employing new technologies. This also helps keep costs down and thus keeping the service more affordable and accessible for the general travelling public. Each bus in the TFL network has been equipped with state of the art GPS enabled automatic vehicle location (AVL) system named iBus\cite{ibusdeployment}. This AVL system has led to improved fleet management and has enabled the creation and improvement of multiple applications \cite{eps354267}.
The system is generating large sets of data both in real-time as well as historical data. This helps, the bus operators and the emergency control room at TFL responsible for maintaining the bus network (CentreComm), to better manage and maintain the smooth operation of the bus network.However there are currently problems for which CentreComm staff are required to manually analyse these data sets in order to discover in which parts of the networks problems are occurring. This is because there is lack of readily available preprocessed information. This is very impractical and time consuming and currently they ultimately rely on individual bus operators and drivers to notify them of possible problems. Once alerted of a possible disruption in the network they can start their own investigation into first verifying what they have been told by the bus drivers/operators and then into finding the cause and the actual severity of the problem. This often could lead to spending time and resources into investigating non existent problems. This is where this project comes in place to address this inefficiency and to propose, implement and evaluate a prototypical tool for real time monitoring of the bus network.

\section{Scope}
The scope of this project is to analyse the current work flow of CentreComm operators and their needs. This has to be followed by designing, implementing and evaluating a prototype which to aid the control room staff. This tool has to work and analyse the data available in real time. The main two problems that are in the scope of this project and need to be solved are:
\begin{itemize}
	\item Analysing and identifying the disruptions in the network in real-time using the available data.
	\item Producing a prioritised list of the disruptions classified using rules gathered during meetings and discussions with the key stakeholders from TFL.
	\item Visualising the calculated list in an appropriate format.
	\item Evaluating the performance of the tool. This is essential as it need to be proven that the results could be trusted.
\end{itemize} 
\section{Aims}
The main aim of this project is to design and implement a real-time visualisation tool which to highlight disrupted routes or parts of the TFL bus network where disruption might happen or have already happened even before the bus drivers or operators have noticed and alerted CentreComm. This could could be subdivided into two smaller aims:
\begin{itemize}
	\item The first one which is independent of the other is to enable the processing of the data generated by the buses in the TFL's bus network. The tool need to be able to analyse the input data sets and calculate and output a list of the disruption that are observed in the network. It has to present information regarding the location (section) in the transport network and their severity.
	\item The second part of the main aim above is to visualise the generated output in an easy to use and understand way.
\end{itemize}
%	1. Are broad statement of the desired outcome, or general intentions of the project
%	2. Emphasize what is to be accomplished (not how it is to be accomplished)
%	3. Address the long-term project outcomes

\section{Objectives}
%	1. Are the steps you are going to take to answer your project questions or specific list of tasks needed to accomplish the goals of the project
%	2. Emphasize how aims are to be accomplished
%	3. Must be highly focused and feasible
%	4. Address the more immediate project outcomes
%	5. Make accurate use of concepts
%	6. Must be sensible and precisely described
%	7. Should read as an individual statement to convey your intentions

\section{Report Structure}
In order to help the reader I have outlined the project structure here. The report would continue in the next chapter by providing the reader with all the background knowledge needed for the rest of the report. This would include brief of background on the current work-flow CentreComm operators follow and its inefficiencies. I will also give background on the iBus system and the data that the tool would need to operate with. I then explore related work that has already been done and how ours differs. This is followed by alternative approaches and models that could be utilised. Afterwards the report focusses on the specific requirements that have been identified and gathered from CentreComm. The report then goes on to discuss the design and the implementation of the proposed system. This is followed then by Chapter 5 and 6 which address testing and evaluation of the prototype. I conclude the report with a summary of what has been achieved and guidance how the work presented in this report could be further developed and improved.