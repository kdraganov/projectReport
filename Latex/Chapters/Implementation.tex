\chapter{Implementation}
This chapter aims to present the reader with explanation of the key implementation aspects. These include major challenges, decisions and problems that have been encountered and taken during the course of this project. I have tried not to go into too much technical details except where this would provide better insight and understanding.

\section{iBus AVL Data}
INCLUDE SAMPLE OF THE DATA IN THE APPENDIX
%Include a sample of the data and explanation of all fields
%Explain what the schedule deviation value means
%Include discussion on how the network is represented and the source of the file - TFL open data
iBus system generate very short telegram messages with the bus location \cite{Hounsell201276}. This information is then processed on a server and more information is calculated and derived.

The data that has been provided by Technical Service Group (TSG) at TFL for this project consists of preprocessed iBus feeds. This feeds currently are being generated every 5 minutes. Each file consists of the following fields:
\begin{enumerate}
\item BusId
\end{enumerate}

The information we are interested is the deviation from the schedule. This is calculated by knowing the expected arrival time of the bus at a stop from the schedule and is compared to the observed time. This value is calculated the same way for both low and high frequency buses(Low frequency buses are supposed to run according to a fixed schedule (e.g. a bus should arrive at stop at predefined time) and usually routes which have less than 5 buses an hour. High frequency bus routes should maintain headways - this meaning a bus should be arriving at stop at predefined intervals (e.g. each 2 minutes)). More about the supplied data for this project would be covered in the requirements section.

Key property of time series data is stationarity. This means that the behaviour of the time series data does not change over time. In our case the data generally speaking the time series data is not stationary. However if we consider short window size it might be possible to treat the data as close to stationary.

\section{Database}
Why have I have decided to use database vs flat file data store. Pros and cons
Why postgresql. Reference to the database model
Explanation and discussion of the database model.

\section{Disruption Engine}
However as other research has pointed out \cite{1251929} calculating travel time as measure for congestion is difficult task and it is very dependable on the environment and its conditions (e.g. weather, time of day, public demand etc.). For this reason and because of the data available this project would not try to measure disruptions by calculating travel time or bus speeds.TFL has provided us with example of the AVL data which among other things contains a the GPS coordinates of the bus at a given point in time and preprocessed deviation from the schedule value. For the rest of this chapter we assume this value is accurately calculated and that we would receive this value for each bus in the network at some regular interval. The provided data is discussed in further details in Chapter 5.

For this project we monitor and measure the schedule deviation value as calculating the congestion is very challenging and still not very well understood in the case of arterial urban traffic.

However from the literature \cite{1251929} we can see that it could be difficult to precisely define what we mean by congestion in a transport network. It seems that congestion could mean various things to different studies and people.

What is the general approach I have taken
%WHAT ARE THE CHALLENGES AND HOW WERE THEY OVERCOMED
	\subsection{Technologies}
	SCALA, XML database connection file, Database is Postregsql
	Why Scala and DB Postgresql.
	\subsection{Bus Network representation}
	What other data is needed?
	\subsection{Monitoring for new feeds}
	
	\subsection{Feed File Processing and observation extraction}
	How is the input processed etc. ...
	
	\subsection{Updating network state}
	Graphics to depict how the algorithm works
		\subsubsection{Applying WMA}
	The important factors to consider are the weights and the period/window size to use. We could also apply exponential smoothing on top of the WMA.
	Problems:
The bus could have started the journey ahead of schedule and thus to intentionally be losing time.
The buses could curtail anywhere on a route without notification
The buses could be diverted (this could be short term or long term) it can also be for few stops or it could be a longer diversion.

		\subsubsection{Applying EMA}
		
		\subsubsection{Detecting disruptions}
	
	\subsection{Problems \& Optimisations}
	
\section{Graphical User Interface}
Include a figure showing the layout. It is responsive etc.
Why Ruby on Rails as front end. Foundation as CSS layout framework as it is lightweight and it enables easy responsive design.
Ajax short polling - as this is only prototype and is supported by all major browsers.


\subsection{Problems}
Data available, data frequency, no knowledge when buses do curtail, not taking into account bus dwell time, bus drivers who are running ahead of schedule could be driving slower on purpose and thus. Tool gives an upper bound of the of the WMA lost time per section. 
