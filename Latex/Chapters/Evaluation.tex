\chapter{Results \& Evaluation}

\section{Evaluation}

Averaging methods: These techniques could be evaluated by calculating the error (the difference between the prediction and the actual value), the squared error and also the the sum of the squared errors (SSE) and respectively the mean of the squared errors (MSE).
The model which minimizes the MSE is the best. It can be shown mathematically that the one that minimizes the MSE for a set of random data is the mean.

Consider metrics for Mean Absolute Difference and Mean Absolute Error. Peak analysis.
Determining the optima Weights and the data window size.
\section{Results}

\subsection{Problems}
Data available, data frequency, no knowledge when buses do curtail, not taking into account bus dwell time, bus drivers who are running ahead of schedule could be driving slower on purpose and thus. Tool gives an upper bound of the of the WMA lost time per section. 


\section{Project Retrospective}
\begin{itemize}
	\item Describe the project approach e.g. Agile incremental development approach. It pros and cons etc. 
	\item What has worked well
	\item What has not worked well
	\item Lessons learnt
\end{itemize}

A project post-mortem, also called a project retrospective, is a process for evaluating the success (or failure) of a project's ability to meet business goals. 

Post-mortems can encompass both quantitative data and qualitative data. Quantitative data include the variance between the hours estimated for a project and the actual hours incurred. Qualitative data will often include stakeholder satisfaction, end-user satisfaction, team satisfaction, potential re usability and perceived quality of end-deliverables.