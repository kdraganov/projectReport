\chapter{Requirements}

In this subsection I have introduced and formalised the current requirements.
This is an important step of any project as it formalises the problems that the
project is trying to address. It also allows to be used as a measure for evaluating
the success of the project once it has been completed. The requirements
presented below could evolve during the progressing of the project however
they are expected to remain mostly the the as defined.
	\section{User Requirements}
	The user requirements provide a list of the functionalities that the user is
expecting to be able to see in the end product. The system is expected to be
able carry out and achieve the following actions and tasks:
• The tool must be able to produce a prioritised list of the disruption in
the bus network that it has knowledge of.
• The tool must be prioritising the disruptions according to the user defined
rules (these are still discussed and gathered from the user)
The tool must be updating this list of disruptions whenever there is more
data. This should happen as real-time as possible.
• The tool must be able to provide more details regarding a disruption (e.g.
which section and which routes are affected and what is the severity).
• The user must be able to interact with the system in order to lower or
increase the priority of a given disruption (even ignore one).
Based on the above list of requirements the use case diagram below presents
the way in which users (actors) interact with the tool (system).
	\section{Functional Requirements}
	System requirements are the specification of how the system behaves. These
are requirements are built using the user requirements as an input. They are
details of what the system (tool) should be able to accomplish technically.
• The system must be able to read and process CSV (comma-separated
values) files.The system must be able to calculate and perform time-series analysis
on the data producing list of the disruption in the network according to
configurable threshold parameters (e.g. delay of more than 20 minutes
in a given section of a route).
• The system must be able to update it self as near real-time as possible
without user interaction or request.
• The system must be able to visualise the output as list.
• The system must be able to expand each list item and display further
details.
The functional requirement require us to deal with CSV file which are currently
updated each 5 minutes and pushed each 15 minutes. The technical
group at TFL has agreed that it would not be problem accessing the 5 minute
snapshots. However going more real-time than 5 minutes probably would be
left for post deployment as it would require a formal change request to be raised
which costs time and resources. The system must be capable of updating itself
as near real-time as possible even though for the moment we work with 5 to
15 minutes update intervals.
The CSV files that the tool would have to work contains number of fields
including:
• Unique vehicle identifier.
• Time when the data has been received.
• The type of the bus trip (e.g. normal trip with passengers, trip to depot
etc.).
• The route number.
• The last stop.
• Deviation from the schedule as described in the background section above.
11
• Longitude and latitude.
There a number of other fields which are not interesting with respect to this
project. The data is unordered and there are multiple files produced one for
each bus operator company. The tool should be able to aggregate all the data
and analysis it and treat it as a single network.