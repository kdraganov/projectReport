\chapter{Implementation}
This chapter aims to present the reader with explanation of the key implementation aspects. These include major challenges, decisions and problems that have been encountered and taken during the course of this project. I have tried not to go into too much technical details except where this would provide better insight and understanding.

\section{iBus AVL Data}
%Include a sample of the data and explanation of all fields
%Explain what the schedule deviation value means
%Include discussion on how the network is represented and the source of the file - TFL open data
iBus system generate very short telegram messages with the bus location \cite{Hounsell201276}. This information is then processed on a server and more information is calculated and derived.

The data that has been provided by Technical Service Group (TSG) at TFL for this project consists of preprocessed iBus feeds. This feeds currently are being generated every 5 minutes. Each file consists of the following fields:
\begin{enumerate}
\item BusId
\end{enumerate}

The information we are interested is the deviation from the schedule. This is calculated by knowing the expected arrival time of the bus at a stop from the schedule and is compared to the observed time. This value is calculated the same way for both low and high frequency buses(Low frequency buses are supposed to run according to a fixed schedule (e.g. a bus should arrive at stop at predefined time) and usually routes which have less than 5 buses an hour. High frequency bus routes should maintain headways - this meaning a bus should be arriving at stop at predefined intervals (e.g. each 2 minutes)). More about the supplied data for this project would be covered in the requirements section.

\section{Database}
Explanation and discussion of the database model.

\section{Disruption Engine}
	\subsection{Technologies}
	SCALA, XML database connection file, Database is Postregsql
	Why Scala and DB Postgresql.
	\subsection{Input}
	How is the input processed etc. ...
	\subsection{Bus Network representation}
	
	\subsection{Applying WMA}
	The important factors to consider are the weights and the period/window size to use. We could also apply exponential smoothing on top of the WMA.
	Problems:
The bus could have started the journey ahead of schedule and thus to intentionally be losing time.
The buses could curtail anywhere on a route without notification
The buses could be diverted (this could be short term or long term) it can also be for few stops or it could be a longer diversion.

	\subsection{Applying EMA}
	
	\subsection{Problems \& Optimisations}
	
\section{Graphical User Interface}
Include a figure showing the layout. It is responsive etc.
Why Ruby on Rails as front end. Foundation as CSS layout framework as it is lightweight and it enables easy responsive design.
Ajax short polling - as this is only prototype and is supported by all major browsers.




