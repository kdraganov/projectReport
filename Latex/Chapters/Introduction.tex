%This is one of the most important components of the report. It should begin with a clear statement of what the project is about so that the nature and scope of the project can be understood by a lay reader. It should summarise everything that you set out to achieve, provide a clear summary of the project's background and relevance to other work, and give pointers to the remaining sections of the report, which will contain the bulk of the technical material \cite{einstein}.
\chapter{Introduction}
%\section{Motivation}
London bus network is one of the largest and most advanced bus networks in the world. It is responsible for more than 2.4 billion passenger journeys a year \cite{TFL1}. The constant population growth of England's capital has been also driving the expansion and improvement of the transport networks across the city \cite{TFL2}.This leads to more pressure being put on the infrastructure which includes not only the road network and the bus fleet, but on the technological systems that aid its operation as well.
%The capital's bus network is continually expanding along with the city's population \cite{TFL2}.

Transport for London (TFL) is in charge of its operation and its bus network is recognised as one of the top in the world in terms or reliability, affordability and cost-effectiveness \cite{TFL1}. Maintaining such a large scale network requires careful planing and monitoring. Being able to operate such a high reliability service 24/7 364 days in the year, requires employing new technologies. This also helps keep costs down and therefore makes the service more affordable and accessible for the general travelling public. 

Each bus in the TFL network is equipped with state of the art GPS enabled automatic vehicle location (AVL) system named iBus\cite{ibusdeployment}. This AVL system has led to improved fleet management and has enabled the creation and improvement of multiple applications \cite{eps354267}. The system is generating large sets of both real-time and historical data, which aids the bus operators and the emergency control room at TFL, responsible for maintaining the bus network (CentreComm), to better manage and maintain the smooth operation of the bus network. This includes both planning for future demand and growth, as well as emergencies and innervation during the daily operation of the network. 

However, there are still some situations and problems which require CentreComm staff to carry out manual analysis of the available data. This means that there is lack of readily available preprocessed information. Having to manually monitor thousands of buses continuously is very impractical. That is the reason why currently CentreComm operators rely heavily on individual bus operators and drivers to alert them of possible problems. Once alerted about a potential disruptions in the network, they (CentreComm) can start their own investigation - first verifying what they have been told by the bus drivers/operators and then into finding the cause and the actual severity of the problem. This could often result in time and resources being spent (wasted) on investigating non existent problems. Worse, it often leads to time and resources being spend on investigating and dealing with problems of less importance than others, because of wrong interpretation (exaggeration) by bus drivers and/or operators involved in these situations. This project tries to address these inefficiencies and to propose a prototypical tool for real-time monitoring of the bus delays in London's bus network.

%This is where this project comes in place to address this inefficiency and to propose, implement and evaluate a prototypical tool for real time monitoring of the bus network.
%This way of operation however often leads to some less important problems being treated with higher priority over some actually more important ones. 
%Having to spend time to manually work through hundreds lines of information is very impractical, time consuming and error prone.
%However there are currently problems for which CentreComm staff are required to manually analyse these data sets in order to discover in which parts of the networks problems are occurring.
%This is because there is lack of readily available preprocessed information. 
%This is very impractical and time consuming and currently they ultimately rely on individual bus operators and drivers to notify them of possible problems. 
%Once alerted of a possible disruption in the network they can start their own investigation first verifying what they have been told by the bus drivers/operators and then into finding the cause and the actual severity of the problem. This often could lead to spending time and resources into investigating non existent problems. This is where this project comes in place to address this inefficiency and to propose, implement and evaluate a prototypical tool for real time monitoring of the bus network.

%This seems an odd mix. Some of these are product requirements, 
%some seem more like tasks that you will need to do to satisfy these requirements. Perhaps worth restructuring? Think about the overall aim, objectives supporting this aim, and finally actions to take to achieve the objectives.

%Project scope is the part of project planning that involves determining and documenting a list of specific project goals, deliverables, tasks, costs and deadlines.
\section{Scope}
The scope of this project is to analyse the current work flow of CentreComm operators and their needs. The main goal is to design and implement a prototype which to automate and improve the work flows currently in place. This tool has to work and analyse the data that has been made available by TFL. This analysis is required to happen in real time as more data is being made available. The reason for this being that it would be used as an objective source of information for the delays in the bus network at each point in time. In addition to this, the project needs to perform analysis of what visualisation will be useful, suitable and usable for the output.

%The four main problems that are in the scope of this project and need to be solved are:
%\begin{itemize}
%	\item Analysing and identifying the disruptions in the bus network in real-time using the provided data.
%	\item Producing a prioritised list of the disruptions classified using rules and heuristics gathered during meetings and discussions with the key stakeholders from TFL.
%	\item Visualising the calculated list in an appropriate format.
%	\item Evaluating the performance of the tool. This is essential as it need to be proven that the results could be trusted.
%\end{itemize} 

\section{Aims}
The main aim of this project is to design and implement a real-time visualisation tool which will be used highlight disrupted routes or parts of the TFL bus network which experience delays. Potentially, the system could alert (be proactive) of possible delays even before the bus drivers or operators have noticed and contacted CentreComm for assistance. This aim could be subdivided into two smaller aims:
\begin{itemize}
	\item The first one, which is independent of the other, is to enable the processing of the data generated by the buses in the TFL's bus network. The tool needs to be able to analyse the input data sets, calculate and output a list of the disruption that are observed in the network. It has to present information regarding the location (route section) in the transport network and their severity.
	\item The second part of the main aim is to visualise the generated output in a way that is easy to use and understand. It is also important to note that the visualisation should be capable of updating itself whenever the list of delays have changed. This needs to happen in real time as well.
\end{itemize}
%In addition to the above aims the project has to evaluate the tool that is developed.

%	1. Are broad statement of the desired outcome, or general intentions of the project
%	2. Emphasize what is to be accomplished (not how it is to be accomplished)
%	3. Address the long-term project outcomes
\section{Objectives}
The objectives that have been followed in order to successfully meet the above stated aims are:
\begin{itemize}
	\item Obtain an in depth understanding of the problem and current work-flows that are in place at CentreComm.
	\item Research similar work in the literature that has already been done and how it relates to our problem.
	\item Obtain samples of the available data and gain an in-depth understanding of it (e.g. what it means).
	\item Gather, analyse and formalise user requirements during discussions and meetings with CentreComm staff and stakeholders.
	\item Design and develop initial prototype, based on the output from the above objective, which is to be further refined and improved upon obtaining feedback from TFL.
	\item Test and evaluate that the tool works according to the user requirements and the design specifications.
\end{itemize}
%	1. Are the steps you are going to take to answer your project questions or specific list of tasks needed to accomplish the goals of the project
%	2. Emphasize how aims are to be accomplished
%	3. Must be highly focused and feasible
%	4. Address the more immediate project outcomes
%	5. Make accurate use of concepts
%	6. Must be sensible and precisely described
%	7. Should read as an individual statement to convey your intentions

%Following the completion of the project, it is expected that each of these objectives are fulfilled and the main project goal is achieved. The following sections will explain the project further and expand on the aims and objectives by conducting research and defining the system’s requirements and specification.

\section{Report Structure}
In order to help the reader, here I have outlined the project structure. The report will continue in the next chapter by providing the reader with a detailed background knowledge needed for the rest of the report, as well as an in-depth review of the related work that is found in the literature. This will include brief background on the current work-flow CentreComm operators follow and its inefficiencies. I will also give background on the iBus system and the data that the tool will need to operate with. In the subsequent chapter, I will then explore related work that has already been done and how ours differs. This is followed by alternative approaches and models that could be utilised. Afterwards the report focusses on the specific requirements (Chapter 3) that have been identified and gathered from CentreComm. The report then goes on to outline the design (Chapter 4) and the implementation (Chapter 5) of the proposed system,  followed by Chapters 6 and 7 which address testing and evaluation of the prototype respectively. I conclude the report with a summary of what has been achieved and guidance how the work presented in this thesis could be further developed and improved.