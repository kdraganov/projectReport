\chapter{Conclusion \& Future Work}

The project's conclusions should list the key things that have been learnt as a consequence of engaging in your project work. For example, ``The use of overloading in C++ provides a very elegant mechanism for transparent parallelisation of sequential programs'', or ``The overheads of linear-time n-body algorithms makes them computationally less efficient than $O(n \log n)$ algorithms for systems with less than 100000 particles''. Avoid tedious personal reflections like ``I learned a lot about C++ programming...'', or ``Simulating colliding galaxies can be real fun...''. It is common to finish the report by listing ways in which the project can be taken further. This might, for example, be a plan for turning a piece of software or hardware into a marketable product, or a set of ideas for possibly turning your project into an MPhil or PhD.

\section{Conclusion}

\section{Future Work}
It could use peak detection to make distinction between incidents and congestion.
Data from other source could be used (taxis AVL, couriers services AVL) etc.
Historical data could be employed in order to make further analysis and correlations with weather data, time or the day/week/year etc.
Increasing the frequency of the data means that we could make use of the actual geo location information in order to calculate and monitor the bus speeds rather than the preprocessed schedule deviation value.

\section{Project Retrospective}
\begin{itemize}
	\item Describe the project approach e.g. Agile incremental development approach. It pros and cons etc. 
	\item What has worked well
	\item What has not worked well
	\item Lessons learnt
\end{itemize}

A project post-mortem, also called a project retrospective, is a process for evaluating the success (or failure) of a project's ability to meet business goals. 

Post-mortems can encompass both quantitative data and qualitative data. Quantitative data include the variance between the hours estimated for a project and the actual hours incurred. Qualitative data will often include stakeholder satisfaction, end-user satisfaction, team satisfaction, potential re usability and perceived quality of end-deliverables.