\chapter{Evaluation}

\section{Problems}
Data available, data frequency, no knowledge when buses do curtail, not taking into account bus dwell time, bus drivers who are running ahead of schedule could be driving slower on purpose and thus. Tool gives an upper bound of the of the WMA lost time per section. 
\section{System Evaluation}

\section{Usability}

\subsection{Project Retrospective}
A project post-mortem, also called a project retrospective, is a process for evaluating the success (or failure) of a project's ability to meet business goals. 

Post-mortems can encompass both quantitative data and qualitative data. Quantitative data include the variance between the hours estimated for a project and the actual hours incurred. Qualitative data will often include stakeholder satisfaction, end-user satisfaction, team satisfaction, potential re usability and perceived quality of end-deliverables.