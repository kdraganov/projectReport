\chapter{Implementation}

\section{Data}
%Include a sample of the data and explanation of all fields
%Explain what the schedule deviation value means
%Include discussion on how the network is represented and the source of the file - TFL open data
iBus system generate very short telegram messages with the bus location \cite{Hounsell201276}. This information is then processed on a server and more information is calculated and derived.

The data that has been provided by Technical Service Group (TSG) at TFL for this project consists of preprocessed iBus feeds. This feeds currently are being generated every 5 minutes. Each file consists of the following fields:
\begin{enumerate}
\item BusId
\end{enumerate}

The information we are interested is the deviation from the schedule. This is calculated by knowing the expected arrival time of the bus at a stop from the schedule and is compared to the observed time. This value is calculated the same way for both low and high frequency buses(Low frequency buses are supposed to run according to a fixed schedule (e.g. a bus should arrive at stop at predefined time) and usually routes which have less than 5 buses an hour. High frequency bus routes should maintain headways - this meaning a bus should be arriving at stop at predefined intervals (e.g. each 2 minutes)). More about the supplied data for this project would be covered in the requirements section.

\section{Disruption Engine}
	\subsection{Input}
	How is the input processed etc. ...
	\subsection{Bus Network representation}
	
	\subsection{Applying WMA}
	
	\subsection{Applying EMA}
	
	\subsection{Problems \& Optimisations}
	
\section{User Interface}
