\chapter{Testing}
The disruption engine consists of a number of individual components and algorithms.
These are tested using a number of unit tests. Due to the nature of
the engine, stress tests were carried out to test it for any memory leaks and
performance issues. Evaluation of the output of the disruption engine will be
performed by first carrying out further literature review to find some guidance
on what window for the data to consider to use and what weights to use for
optimal results. This will be followed by running the system on a given set
of data (e.g. a week worth of AVL data) and comparing the output with the
actual state of the network during this period.

The user interface consists of a simple web application capable of displaying
list of disruptions. Testing and evaluation of this web application will be
performed as user testing. As I mentioned above, some feedback and problems
were identified which will be addressed and fixed. Follow up user tests will be
carried out by giving access to friends and family to the web site to use and
give feedback on. This should give reasonable confidence in the correctness of
the user interface as there is no complex logic incorporated in the web front
end application.
\section{Unit Testing} 
What is it?
Why was needed?
How was it performed?

\section{Functional Testing}
Focus on the user interface.
What is it?
Why was needed?
How was it performed?

\section{Integration Testing}
What is it?
Why was needed?
How was it performed?

\section{Stress Testing}
Why was this needed?
How was it carried out? - Results/proof