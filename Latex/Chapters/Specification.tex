\chapter{Specification}
In this chapter I have introduced and formalised the user and functional requirements.
This is an important step of any project, especially computer science project, as it formalises the problems that the project is trying to address as outlined by the project aims and objectives in Chapter 1. It also allows to be used as a measure for evaluating the success of the project once it has been completed. The requirements presented below have evolved and have been refined throughout the project lifetime in response to feedback from and discussions carried out with, the key stakeholders.

\section{User Requirements}
The user requirements provide a list of the functionalities that the user(s) expect(s) to be able to perform and see/obtain the according results, in the end product system. These are what is expected from the system, but are not concerned how they are designed or implemented. The main user requirements are listed as follows:

\begin{enumerate}
	\item The tool must be able to produce a prioritised list of the disruptions in
the bus network that it has knowledge of.
	\item The tool must be prioritising the disruptions according to the user defined
rules (these are still discussed and gathered from the user).
	\item The tool must be updating this list of disruptions whenever there is more
data. This should happen as real-time as possible.
	\item The tool must be able to provide detailed information for every detected disruption. This has to include the specific section and route that are affected and its severity.
	\item The user must be able to interact with the system in order to lower or
increase the priority of a given disruption (even ignore one).
\end{enumerate}

\section{Functional Requirements}
These requirements specify in more details what the expected behaviour and functionality of the system/tool is. They are built on top of the user requirements as an input and are detailed list of what the system should be able to accomplish technically. The tool/system must:

\begin{enumerate}
	\item Have an appropriate and useful representation of the bus network in order to be able to monitor and detect problems in it.
	\item Be able to read and process CSV files as this is the primary input of the AVL feed files (more detailed discussion on the exact input and its format is presented in Chapter 5).
	\item Listen/monitor for new incoming data and process it in as close as possible to real time.
	\item Be able to update itself whenever new data is detected and processed.
	\item Be able to run without intervention 24/7.
	\item Keep track of the disruptions detected and track how they evolve and develop.
	\item Be able to keep information for a given window of time (e.g. data feeds from the last 2 hours).
	\item Be able to output a prioritised list of disruptions.
	\item Visualise the generated output appropriately.
	\item The system should be compatible with and accessible from every CentreComm staff's computer. Extension of this requirement is that it should be accessible from other teams and groups inside TFL.
	% It should be compatible to run under Firefox or Internet Explorer as this are the main browsers used by CentreComm/TFL staff.
	\item Display on request detailed information for the respective disruption. This should include a graph representing the route/section average disruption time.
	\item Be easy to deploy, configurable and maintainable.
\end{enumerate}